\usepackage[lmargin=2cm,rmargin=2cm,top=1.5cm,bottom=2cm]{geometry}  %paquete parra margenes 
\usepackage[T1]{fontenc} %para simbolos  en  español 
\usepackage[utf8]{inputenc} %para tildes 
\usepackage[spanish,es-tabla]{babel} %Reconosca espanol y tabla 

\usepackage{amsmath} %simbolos matematicas
\usepackage{amssymb, amsfonts,latexsym, cancel} %matematicas
\usepackage{bm} %negritas
\usepackage{float} % para figuras donde queramos 

\usepackage{fancyhdr}%formato al documento
\usepackage{epstopdf}%pes a pdf
\usepackage[colorlinks=true,linkcolor=blue' urlcolor=blue]{hyperref} %para cololoriar link
\usepackage{longtable} % para  tablas largas
\setcounter{MaxMatrixCols}{40} %Para  matrices de hasta 40 colunas
\usepackage{multicol} %varias columnas
\usepackage{subfigure} %colocar varias figuras
\usepackage{titling} %  PARA FORMATO DE TITULO 
\usepackage{anyfontsize}

\usepackage[dvips]{graphicx}
\usepackage{graphicx}
%\usepackage[pdftex]{hyperref}
\usepackage{eurosym}
\usepackage{array,booktabs}
\usepackage{enumerate}
\usepackage{fancyvrb}
\usepackage{listings}
\usepackage{amsfonts}
\usepackage{color}
\usepackage{multirow}


\topmargin -0.1cm
\headsep 0.5cm 
\oddsidemargin 0.4cm   % read Lamport p.163
\evensidemargin -0.4cm  % same as oddsidemargin but for left-hand pages
\textheight=23.5cm
\textwidth=16cm

\usepackage{float} %to put table caption on top
\floatstyle{plaintop}
\restylefloat{table}


\renewcommand{\chaptermark}[1]{\markboth{\textbf{\thechapter. #1}}{}} % Chapter format: N. Name
\renewcommand{\sectionmark}[1]{\markright{\textbf{\thesection. #1}}} % Section format: N.M. Name
\setlength{\belowcaptionskip}{6pt}   % 0.5cm as an example
\lstloadlanguages{[ANSI]C}

% numerate including subsubsection
\setcounter{secnumdepth}{3}
\setcounter{tocdepth}{3}


\newenvironment{nscenter} %used for lines centering
 {\parskip=0pt\par\nopagebreak\centering}
 {\par\noindent\ignorespacesafterend}




\usepackage{color}
\usepackage{graphicx} % for pdf, bitmapped graphics files
\usepackage{epsfig} % for postscript graphics files
\usepackage{mathptmx} % assumes new font selection scheme installed
\usepackage{times} % assumes new font selection scheme installed
\usepackage{amsmath} % assumes amsmath package installed
\usepackage{amssymb}  % assumes amsmath package installed
\usepackage[percent]{overpic}
%\usepackage{enumitem}
\usepackage{stmaryrd}

\DeclareMathOperator*{\argmin}{\mathsf{arg\,min}}
\DeclareMathOperator*{\argmax}{\mathsf{arg\,max}}
\DeclareMathOperator*{\superior}{\mathsf{sup}}

\newcommand{\ud}{\mathrm{d}}
\newcommand{\partder}[2]{\frac{\partial #1}{\partial #2}}
\newcommand{\deriv}[2]{\frac{\ud #1}{\ud #2}}
\newcommand{\R}{\mathbb{R}}
\newcommand{\dA}{\bar A}
\newcommand{\dB}{\bar B}
\newcommand{\dQ}{\bar Q}
\newcommand{\dR}{\bar R}
\newcommand{\dS}{\bar S}
\newcommand{\dP}{\bar P}
\newcommand{\dK}{\bar K}
\newcommand{\dW}{\bar W}
\newcommand{\dL}{\bar L}
\newcommand{\ceil}[1]{\left\lceil #1 \right\rceil}
\newcommand{\w}{\omega}
\newcommand{\tN}{t_N}

\renewcommand{\arraystretch}{0.9} % because \baselinestretch is 1.6667
\renewcommand{\arraycolsep}{1.2pt}